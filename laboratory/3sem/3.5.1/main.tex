\documentclass[12pt]{article}
\usepackage[T2A]{fontenc}
\usepackage[utf8]{inputenc}
\usepackage[english,russian]{babel}
\usepackage[normalem]{ulem}
\usepackage{graphicx}
\usepackage{float}
\usepackage{amsmath}
\usepackage{tikz}
\usepackage{misccorr}
\usepackage{geometry}
\usepackage{caption}
\usepackage{booktabs}
\usepackage{wrapfig}
\usepackage{siunitx}
\geometry{left=1cm}
\geometry{right=1cm}
\geometry{bottom=3cm}
\geometry{top=3cm}
\DeclareGraphicsExtensions{.pdf,.png,.jpg}
\begin{document}
\begin{titlepage}
  \begin{center}
  \large
    Московский физико-технический институт (государственный университет)
    \vfill
 
    \textsc{Лабораторная работа}\\[5mm]
  
    {\LARGE 3.5.1. Исследование плазмы газового заряда.}
  \vfill
    Чумаченко Артем Дмитриевич \\
    2 курс, группа 790
\end{center}
\vfill
 
\begin{center}
  Долгопрудный, 2018г.
\end{center}
\end{titlepage}


\section{Практическая часть}

\begin{enumerate}
	\item Построим вольт-амперную характеристику разряда $U = f(I_p)$. По наклону прямой определим максимальное дифференциальное сопротивление разряда.
	
	\begin{tabular}{|r|l|l|l|l|l|l|l|l|}
	\hline
	$I, \si{\ampere} \cdot 10^{-3}$ & 0.52 & 1.04 & 1.56 & 2.12 & 3.12 & 3.64 & 4.16 & 4.68 \\ \hline
	$U, \si{\volt}$ & 34.9 & 33.8 & 32.7 & 30.9 & 28.7 & 27.6 & 26.6 & 26.0 \\
	    \hline
	\end{tabular}
	$\sigma_I = 0.05 \cdot 10^{-3} \si{\ampere}$, $\sigma_U = 0.4 \si{\volt}$
	
	\begin{figure}[htb]
	    \centering
	    \includegraphics[scale=0.6]{img/VAC.png}
	    \label{fig:VAC}
	\end{figure}
	
	\begin{tabular}{|c|c|c|c|}
	\hline
	     $R, \si{\ohm}$ & $\sigma_{\text{ins}}$ & $\sigma_{\text{std}}$ & $\sigma$ \\ \hline
	     2230 & 80 & 77 & 120 \\ \hline
	     $\varepsilon$ & $3.5\%$ & $3.5\%$ & $5.2\%$ \\ \hline
	\end{tabular}
	\newpage
	
	\item Построим семейство отцентрированных зондовых зарядов на одном листе. Определим температуру электронов по формуле:
	
	\begin{equation}
	    k T_e = \dfrac{1}{2} \dfrac{e I_{\text{iH}}}{\dfrac{d I}{d U }\Bigr|_0}
	\end{equation}
	
	\begin{tabular}{|l|r|r}
	\toprule
	$I_p, \si{\milli \ampere}$ & 5.2 & \\
    \toprule
    {} &         $I, \si{\micro \ampere}$ & $U, \si{\volt} $\\
    \midrule
    0  & -101 & -25.00 \\
    1  & -98 & -22.05 \\
    2  & -95 & -19.01 \\
    3  & -92 & -16.05 \\
    4  & -87 & -13.02 \\
    5  & -78 & -10.06 \\
    6  & -70 &  -8.10 \\
    7  & -58 &  -6.00 \\
    8  & -44 &  -3.96 \\
    9  & -31 &  -2.12 \\
    10 & -12 &   0.00 \\
    11 &  12 &   0.00 \\
    12 &  30 &   2.09 \\
    13 &  44 &   4.00 \\
    14 &  57 &   6.08 \\
    15 &  68 &   8.13 \\
    16 &  75 &  10.01 \\
    17 &  83 &  13.00 \\
    18 &  88 &  16.00 \\
    19 &  91 &  19.09 \\
    20 &  93 &  22.10 \\
    21 &  95 &  24.89 \\
    \bottomrule
    \end{tabular}
    \begin{tabular}{|l|r|r|}
    \toprule
    $I_p, \si{\milli \ampere}$ & 3.12 & \\
    \toprule
    {} &         $I, \si{\micro \ampere}$ & $U, \si{\volt} $\\
    \midrule
    0  &  54 &  25.00 \\
    1  &  52 &  22.04 \\
    2  &  51 &  19.01 \\
    3  &  49 &  16.05 \\
    4  &  47 &  13.00 \\
    5  &  43 &  10.60 \\
    6  &  38 &   8.05 \\
    7  &  32 &   6.08 \\
    8  &  24 &   4.13 \\
    9  &  13 &   2.08 \\
    10 &  1 &  -0.00 \\
    11 & -13 &  -2.02 \\
    12 & -24 &  -4.01 \\
    13 & -33 &  -6.10 \\
    14 & -40 &  -8.14 \\
    15 & -45 & -10.00 \\
    16 & -49 & -13.12 \\
    17 & -51 & -16.05 \\
    18 & -53 & -19.10 \\
    19 & -55 & -22.12 \\
    20 & -56 & -25.04 \\
    & &  \\
    \bottomrule
    \end{tabular}
    \begin{tabular}{l|r|r|}
    \toprule
    $I_p, \si{\milli \ampere}$ & 1.56 & \\
    \toprule
    {} &         $I, \si{\micro \ampere}$ & $U, \si{\volt} $\\
    \midrule
    0  & -28.6 & -25.04 \\
    1  & -27.6 & -22.06 \\
    2  & -26.6 & -19.05 \\
    3  & -25.5 & -16.01 \\
    4  & -24.3& -13.03 \\
    5  & -22.2 & -10.12 \\
    6  & -19.8 &  -8.13 \\
    7  & -16.3 &  -6.03 \\
    8  & -11.8 &  -4.03 \\
    9  & -6.3 &  -2.03 \\
    10 & -0.14 &  -0.01 \\
    11 &  0.66 &   0.05 \\
    12 &  6.7 &   2.01 \\
    13 &  11.8 &   3.96 \\
    14 &  15.5 &   6.03 \\
    15 &  19 &   8.03 \\
    16 &  21.1 &  10.03 \\
    17 &  23 &  13.13 \\
    18 &  24.1 &  15.97 \\
    19 &  25.1 &  19.18 \\
    20 &  26 &  22.18 \\
    21 &  26.8 &  24.70 \\
    \bottomrule
    \end{tabular}
    
    $\sigma_I = 0.5 \si{\micro\ampere}$, $\sigma_U = 0.05 \si{\volt}$
    
    \newpage
	
	\begin{figure}[!htbp]
	    \centering
	    \includegraphics[scale=0.4]{img/zond.png}
	    \label{fig:zond}
	\end{figure}
	\begin{figure}[!htbp]
	\begin{minipage}[c]{0.45\textwidth}
	    \begin{tabular}{|c|c|c|c|c|}
    	\hline
    	     $I, \si{\milli\ampere}$ & $kT, eV$ & $\sigma_{\text{ins}}$ & $\sigma_{\text{std}}$ & $\sigma$ \\ \hline
    	     5.2 & 2.3 & 0.02 & 0.65 & 0.7 \\ \hline
    	     $\varepsilon$ &  & $1\%$ & $28\%$ & $28\%$ \\ \hline
    	     3.12 & 3.2 & 0.05 & 0.21 & 0.21 \\ \hline
    	     $\varepsilon$ &  & $2\%$ & $7\%$ & $7\%$ \\ \hline
    	     1.56 & 3.0 & 0.09 & 0.11 & 0.14 \\ \hline
    	     $\varepsilon$ &  & $3\%$ & $4\%$ & $5\%$ \\ \hline
    	\end{tabular}
	\end{minipage}
	\begin{minipage}[c]{0.45\textwidth}
	\includegraphics[scale=0.5]{img/kT.png}
	\end{minipage}
	\end{figure}
	
	\newpage
	
	Определим концентрацию электронов по формуле Бома:
	
	\begin{equation}
	    n_e = \dfrac{I_{\text{iH}}}{0.4 e S} \sqrt{\dfrac{m_e}{2kT}}
	\end{equation}
	$S = \pi \cdot d \cdot l = \pi \cdot 0.9 \si{\milli \metre} \cdot 5.2 \si{\milli \metre}$.
	
	\begin{tabular}{|c|c|c|}
	\hline
	     $I, \si{\milli\ampere}$ & $n, \si{\metre}^{-3} \cdot 10^{16}$ & $\sigma$ \\ \hline
	     5.2 & 8 & 3 \\ \hline
	     3.12 & 3.8 & 0.5 \\ \hline
	     1.56 & 1.83 & 0.2 \\ \hline
	\end{tabular}
	$\sigma = \sqrt{1/4 \cdot \sigma_{k T}^2 + \sigma_{I_{iH}}^2}$, $\sigma_{I_{iH}} = \sqrt{\sigma_{\text{ins}}^2 + \sigma_{\text{std}}^2}$
	
	\begin{figure}[htb]
	    \centering
	    \includegraphics[scale=0.5]{img/ne.png}
	\end{figure}
	
	\newpage
	\item Рассчитаем плазменную частоту колебаний электронов и дебаевский радиус:
	
	\begin{equation}
	    \omega_p = \sqrt{\dfrac{4 \pi n_e e^2}{m_e}},
	\end{equation}
	\begin{equation}
	    r_D = \sqrt{\dfrac{k T_i}{4 \pi n_i e^2}}.
	\end{equation}
	
	\begin{tabular}{|c|c|c|}
	\hline
	     $I, \si{\milli\ampere}$ & $w_p, \si{\second}^{-1} \cdot 10^9$ & $\sigma$ \\ \hline
	     5.2 & 16 & 3 \\ \hline
	     3.12 & 11 & 0.7 \\ \hline
	     1.56 & 7.6 & 0.5 \\ \hline
	\end{tabular}
	\begin{tabular}{|c|c|c|}
	\hline
	     $I, \si{\milli\ampere}$ & $r_D, \si{\centi \metre} \cdot 10^{-4}$ & $\sigma$ \\ \hline
	     5.2 & 39 & 7 \\ \hline
	     3.12 & 68 & 4 \\ \hline
	     1.56 & 95 & 6 \\ \hline
	\end{tabular}
	$\sigma_{\omega_p} = \omega_p \varepsilon_{n_e} / 2$, $\sigma_{r_D} = r_D \sqrt{\varepsilon_{kT_i}^2 / 4 + \varepsilon_{n_i}^2 / 4}$
	
	Убедимся, что число частиц $N_D >> 1$:
	
	\begin{equation}
	    N_D = n_i \dfrac{4}{3} \pi r_D^3.
	\end{equation}
	
	\begin{tabular}{|c|c|}
	\hline
	     $I, \si{\milli\ampere}$ & $N_D, 10^3$ \\ \hline
	     83.5 & $21 $ \\ \hline
	     200.2 & $50 $ \\ \hline
	     265.3 & $66 $ \\ \hline
	\end{tabular}
	
	И определим относительное число ионов:
	
	\begin{tabular}{|c|c|c|}
	\hline
	     $I, \si{\milli\ampere}$ & $\alpha, 10^{-10}$ & $\sigma$ \\ \hline
	     5.2 & 33 & 9 \\ \hline
	     3.12 & 15 & 1 \\ \hline
	     1.56 & 7.6 & 0.4 \\ \hline
	\end{tabular}

\end{enumerate}

\section{Вывод}

\begin{tabular}{|c|c|c|c|c|c|c|}
\hline
     $R, \si{\ohm}$ & $I_p, \si{\milli \ampere}$ & $kT_e, eV$ & $n_e, \si{\metre}^{-3}  \cdot 10^{16}$ &
     $\omega_p, \si{\second}^-1$ & $r_D, \si{\centi \metre} \cdot 10^{-4}$ & $<N_D>, 10^3$ \\ \hline
     $(2230 \pm 120)$ & 5.2 & $(2.3 \pm 0.7)$ & $(8 \pm 3)$ & $(16 \pm 3)$ & $(39 \pm 7)$ & 21 \\ \hline
      & 3.12 & $(3.2 \pm 0.21)$ & $(3.8 \pm 0.5)$ & $(11 \pm 0.7)$ & $(68 \pm 4)$ & 50 \\ \hline
      & 1.56 & $(3.0 \pm 0.14)$ & $(1.83 \pm 0.2)$ & $(7.6 \pm 0.5)$ & $(95 \pm 6)$ & 66 \\ \hline
\end{tabular}

\end{document}

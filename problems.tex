\documentclass[12pt, a4paper, usenames]{article}
\usepackage[top=15mm,left=15mm,headheight=78pt]{geometry}
\usepackage[utf8]{inputenc}
\usepackage{amssymb, amsmath, multicol}
\usepackage[russian]{babel}

\usepackage{graphicx}
\graphicspath{ {images/} }

\usepackage[shortcuts,cyremdash]{extdash}
\usepackage{wrapfig}
\usepackage{floatflt}
\usepackage{lipsum}
\usepackage{mathrsfs} % буква для обозначения ЭДС
\usepackage{ifthen}
\usepackage{wallpaper}
\usepackage{color}
\usepackage{setspace}


\renewcommand{\tg}{\mathop{\mathrm{tg}}\nolimits}
\renewcommand{\ctg}{\mathop{\mathrm{ctg}}\nolimits}
\renewcommand{\arctan}{\mathop{\mathrm{arctg}}\nolimits}
\newcommand{\divisible}{\mathop{\raisebox{-2pt}{\vdots}}}
\newcommand*{\hm}[1]{#1\nobreak\discretionary{}%
{\hbox{$\mathsurround=0pt #1$}}{}}
\newcommand{\EDS}{\ensuremath{\mathscr{E}}}
\newcommand{\bbskip}{\bigskip \bigskip}
\parindent=0pt
\textwidth=180mm
\textheight=260mm

\pagestyle{empty}
\begin{document}

\begin{titlepage}

\parbox[b][6cm][t]{50mm}{
\includegraphics[scale=0.8]{back}}
\hfill
\parbox[b][6cm][t]{115mm}{
\begin{center}
\textsc{\Large Московский Физико-Технический Институт\\ (государственный университет)}\\

\vspace{12pt}

\textsc{\Large Межвузовский центр воспитания и развития талантливой молодежи в области естественно-математических наук <<Физтех-центр>>}\\

\textcolor[rgb]{0.0274, 0.4509, 0.7333}{\line(1,0){325}}
\end{center}
}

\vspace*{-40pt}

\vspace{200pt}

\begin{center}
\begin{spacing}{2.0}
{ \LARGE \bfseries 58-Я ВЫЕЗДНАЯ ФИЗИКО-МАТЕМАТИЧЕСКАЯ ОЛИМПИАДА МФТИ}\\
\end{spacing}

\textcolor[rgb]{0.0274, 0.4509, 0.7333}{{ \LARGE \bfseries УСЛОВИЯ ЗАДАЧ}}
\end{center}

\vspace{220pt}

\parbox[b][2cm][t]{30mm}{
\includegraphics[scale=0.14]{mipt}}
\hfill
\parbox[b][2cm][t]{70mm}{
\vspace{10pt}
\linethickness{5pt}
\textcolor[rgb]{0.0274, 0.4509, 0.7333}{\line(1,0){325}}
\begin{center} \hspace{2 cm} Москва 2019 \end{center}
}
\hfill
\parbox[b][2cm][t]{40mm}{
\includegraphics[scale=0.08]{abitu}}

\end{titlepage}




\newpage
\begin{center}
Выездная олимпиада по физике\hfill\textit{\bf 8 класс}
\hrule\medskip
\end{center}

{\bf Ф8.1} Из пунктов $A$ и $B$, расположенных на расстоянии $L=50$~км друг от друга вдоль прямолинейного участка шоссе, одновременно в одном направлении начали двигаться два мотоциклиста. Из пункта $A$ мотоциклист едет по направлению к пункту $B$ со скоростью $V_1= 90$~км/ч, а мотоциклист из пункта $B$~---~со скоростью $V_2 = 60$~км/ч. На каком расстоянии от пункта В мотоциклисты встретятся?

\bigskip

{\bf Ф8.2} Вася прошел $2/3$ пути со скоростью $V_1 = 5$~км/ч, а оставшуюся часть пути пробежал с постоянной скоростью. С какой скоростью бежал Вася, если его средняя скорость на всем пути составила $V_\textrm{ср} = 6$~км/ч?

\bigskip

\parbox[b][2cm][t]{140mm}{{\bf Ф8.3} Однородная доска массой $M$ и длиной $L$ положена на опору и находится в равновесии (см.~рис.). На левом конце доски подвешен груз массой $m$, а на правом~---~груз массой $2m$. Найти расстояние между опорой и правым концом доски.
}
\hfill
\parbox[b][2cm][t]{40mm}{
\includegraphics[scale=0.35]{F83}} 

\bigskip


{\bf Ф8.4} Шарик взвешивают с помощью динамометра. Первый раз шарик взвесили в воздухе, а второй раз в жидкости с плотностью $\rho_\textrm{ж}= 0{,}8$~г/см$^3$. Вес шарика в воздухе оказался в $3$ раза больше, чем вес в жидкости. Определите плотность материала шарика.

\bigskip

{\bf Ф8.5} Деревянная пластина толщиной $h = 3$~см плавает в сосуде с водой. Сверху наливают бензин так, что уровень бензина совпадает с верхней поверхностью пластины. Бензин с водой не перемешиваются. Найти высоту налитого слоя бензина. Плотность воды $\rho_\textrm{в} = 10^3$~кг/м$^3$, плотность дерева $\rho_\textrm{д} = 0{,}8\cdot 10^3$~кг/м$^3$, плотность бензина $\rho_\textrm{б} = 0{,}7\cdot 10^3$~кг/м$^3$.

\bigskip

{\bf Ф8.6} В сосуд налита вода массы $m_\textrm{в} = 5$~кг с температурой $t_\textrm{в} = 70^\circ C$. В воду положили лед массы $m_\textrm{л} = 1$~кг с температурой $t_\textrm{л} = -20^\circ C$. Пренебрегая потерями тепла, найти установившуюся температуру воды в сосуде.

Удельная теплоёмкость воды $c_\textrm{в} = 4{,}2\cdot 10^3$~Дж/(кг$\cdot^\circ C$), удельная теплоёмкость льда $c_\textrm{л} = 2{,}1\hm\cdot 10^3$~Дж/(кг$\cdot^\circ C$), удельная теплота плавления льда $\lambda = 3{,}4\cdot 10^5$~Дж/кг.

\bigskip


\parbox[b][2cm][t]{125mm}{{\bf Ф8.7} Найти сопротивление электрической цепи между точками $A$ и $B$. Сопротивление $R = 1$~Ом.
}
\hfill
\parbox[b][2cm][t]{55mm}{
\includegraphics[scale=0.35]{F87}} 




\newpage
\begin{center}
Выездная олимпиада по физике\hfill\textit{\bf 9 класс}
\hrule\medskip
\end{center}

{\bf Ф9.1} К перекрестку по двум взаимно перпендикулярным шоссейным дорогам движутся равномерно грузовая и легковая автомашины со скоростями $V_1 = 15$~м/с и $V_2 = 20$~м/с соответственно. В некоторый момент времени автомашины находятся от перекрестка на расстояниях $S_1 = 300$~м и $S_2 = 275$~м. Через какое время $T$ расстояние между автомашинами будет наименьшим? 

\bigskip

{\bf Ф9.2} С высокой башни с интервалом $\tau = 1$~с бросают с нулевой начальной скоростью два камня. На каком расстоянии $S$ друг от друга будут находиться камни в тот момент, когда скорость второго камня станет равной $V = 30$~м/с? 

Ускорение свободного падения $g = 10$~м/с$^2$. Силу сопротивления воздуха считайте пренебрежимо малой. 

\bigskip

{\bf Ф9.3} Камень вылетает из метательной машины со скоростью $V_1 = 39$~м/с и через $T = 4{,}2$~с попадает в цель. В этот момент скорость камня $V_2 = 45$~м/с. На каком  расстоянии $L$ по горизонтали от машины находится цель?

Ускорение свободного падения $g = 10$~м/с$^2$. Силу сопротивления воздуха считайте пренебрежимо малой. 

\bigskip

{\bf Ф9.4} По клину массой $M$, находящемуся на гладкой горизонтальной плоскости, скользит шайба массой $m$. Гладкая наклонная плоскость клина составляет с горизонтом угол $\alpha$. Определите величину $P$ силы, с которой шайба действует на клин. Ускорение свободного падения $g$. 

\bigskip

{\bf Ф9.5} Стальной кубик плавает в ртути. Поверх ртути наливают воду так, что она только покрывает кубик. Какова высота $h$ слоя воды? Длина ребра кубика $b = 10$~см,  плотность стали $\rho_1 = 7{,}8$~г/см$^3$, плотность ртути $\rho_2 = 13{,}6$~г/см$^3$,  плотность воды $\rho_3 = 1$~г/см$^3$.

\textit{Примечание.} Параллельность грани куба поверхности воды при плавании обеспечивается незначительными внешними усилиями.

\bigskip

{\bf Ф9.6} В калориметр, содержащий $m_1 = 100$~г льда при $t_1 = 0^\circ С$, наливают $m_2 = 150$~г воды при температуре $t_2 = 50^\circ С$. Определите установившуюся в калориметре температуру $t$.  Удельная  теплоемкость воды $c = 4200$~Дж/(кг$\cdot$ К). Удельная теплота плавления льда $\lambda = 3{,}3\cdot 10^5$~Дж/кг.

\bigskip

\parbox[b][2.7cm][t]{115mm}{{\bf Ф9.7} Для измерения сопротивления $R$ проводника собрана электрическая цепь (см. схему на рис.). Вольтметр $V$ показывает напряжение $U_V = 5$~В. Показание амперметра $A$ равно $I_A = 25$~мА. Найдите величину $R$ сопротивления проводника. Внутреннее сопротивление вольтметра $R_V = 1{,}0$~кОм. 
\medskip}
\hfill
\parbox[b][2.7cm][t]{65mm}{
\includegraphics[scale=0.33]{F97}} 

\bigskip



\newpage
\begin{center}
Выездная олимпиада по физике\hfill\textit{\bf 10 класс}
\hrule\medskip
\end{center}

{\bf Ф10.1} Винт вентилятора в момент начала торможения вращается с угловой скоростью $\omega \hm= 25$~с$^{-1}$ и через $\tau = 10$~с останавливается. Сколько оборотов совершит винт за время торможения? Считайте, что в процессе торможения угловая скорость винта уменьшается равномерно по времени.   

\bigskip

{\bf Ф10.2} Из одной точки одновременно бросают два камня с одинаковыми по величине начальными скоростями. Первый камень брошен под углом $\alpha = 30^\circ$ к горизонту, второй~---~вертикально вверх. Через $\tau = 2$~c после старта камни находятся на расстоянии $S = 60$~м друг от друга. Найдите максимальное расстояние $S_{max}$ между камнями в процессе полета камней. Ускорение свободного падения $g = 10$~м/с$^2$. Силу сопротивления воздуха считайте пренебрежимо малой. 

\bigskip

{\bf Ф10.3} Мешок с песком падает вертикально со скоростью $V = 5$~м/с на массивную тележку, движущуюся горизонтально со скоростью $U = 1{,}5$~м/с. Мешок после удара не подскакивает. При каком наименьшем коэффициенте трения скольжения $\mu$ мешок не будет проскальзывать по тележке после обращения в ноль его вертикальной составляющей скорости? Длительность соударения очень мала. 

\bigskip

{\bf Ф10.4} При какой продолжительности $T$ суток на Земле вес тела на экваторе будет вдвое отличаться от веса этого же тела на полюсе? 

Землю считайте однородным шаром. Радиус Земли $R = 6{,}4\cdot 10^6$~м, ускорение свободного падения у поверхности планеты $g = 10$~м/с$^2$.

\bigskip

{\bf Ф10.5} Найдите массу $m$ оболочки наполненного водородом резинового шарика диаметром $d \hm= 25$~см, свободно плавающего  в воздухе. Воздух и водород находятся при нормальных условиях: $t = 0^\circ C$, $P = 10^5$~Па. Молярная масса водорода  $\mu_1 = 2\cdot 10^{-3}$~кг/моль, молярная масса воздуха $\mu_2 = 29\cdot10^{-3}$~кг/моль.  Универсальная газовая постоянная $R = 8{,}31$~Дж/(моль$\cdot$К). Объем   шара $V$ связан с диаметром $d$ соотношением $V = \dfrac{\pi d^3}{6}$. 

\bigskip

{\bf Ф10.6} Два одинаковых металлических шара расположены на большом расстоянии. Заряд одного из шаров $Q$, другой не заряжен. Проводящий незаряженный шарик последовательно приводят в контакт сначала с заряженным шаром, затем с незаряженным. После двух контактов заряд шарика становится равным $Q/9$. 

Какой заряд $q$ перешел на шарик при первом контакте?

\bigskip

{\bf Ф10.7} Электрическая  цепь (см. схему на рисунке) состоит из очень большого (<<бесконечного>>) числа одинаковых звеньев, содержащих сопротивления $r = 1$~Ом.  К точкам $A$ и $B$ подключают источник постоянного напряжения $U = 1{,}5$~В. Какое количество теплоты $Q$ будет ежесекундно выделяться в цепи?

\includegraphics[scale=0.5]{F107}

\bigskip

\newpage
\begin{center}
Выездная олимпиада по физике\hfill\textit{\bf 11 класс}
\hrule\medskip
\end{center}

{\bf Ф11.1} Мяч лежит на горизонтальной поверхности земли на расстоянии $L= 4$~м от вертикального забора высотой $H= 1{,}5$~м. Мальчик сообщает мячу скорость под углом к горизонту в сторону забора. В результате мяч перелетает через забор, почти касаясь его на максимальной высоте своего полета.
\begin{itemize}
\item[1)] Найти время $\tau$ продолжительности полета мяча от его вылета до падения на землю за забором.
\item[2)] Найти начальную скорость $V$, с которой мяч вылетел с поверхности земли.
Ускорение свободного падения считать равным $g = 10$ м/с$^2$.
\end{itemize}

\bigskip


\parbox[b][3.4cm][t]{130mm}{{\bf Ф11.2} Бруски массами $m$ и $2m$ связаны легкой нитью, перекинутой через блок. Блок укреплен на тележке (см. рис.). Верхняя горизонтальная поверхность тележки гладкая, коэффициент трения между вертикальной поверхностью тележки и бруском массой $m$ равен $\mu = 0{,}5$. С каким минимальным горизонтальным ускорением $a$ надо двигать тележку, чтобы брусок массой $m$ поднимался вверх? Массой блока и трением в его оси пренебречь.}
\hfill
\parbox[b][3.4cm][t]{50mm}{
\includegraphics[scale=0.35]{F112}} 

\bigskip

{\bf Ф11.3} Идеальный одноатомный газ в количестве $\nu$ (моль) расширяется от температуры $T_1\hm=T$ до температуры $T_2=1{,}2T$ в процессе с прямо пропорциональной зависимостью давления от объема. Далее газ нагревают изохорически до температуры $T_3=1{,}6T$. Какое количество теплоты получил газ во всем процессе?

\bigskip

{\bf Ф11.4} Плоская катушка из $n=7$ витков находится в однородном магнитном поле с индукцией $B = 0{,}01$~Тл, направленной перпендикулярно плоскости витков катушки. Катушка замкнута на гальванометр. Катушку выносят из магнитного поля. Какой заряд пройдет через гальванометр? Площадь одного витка $S = 2$~см$^2$. Сопротивление витков катушки, подводящих проводов и гальванометра $R = 4$~Ом. 

\bigskip

\parbox[b][4cm][t]{135mm}{{\bf Ф11.5} Параметры идеальных элементов цепи указаны на схеме (см. рис.). Ключ замкнут, режим в цепи установился. Какое количество теплоты выделится в цепи после размыкания ключа?}
\hfill
\parbox[b][4cm][t]{45mm}{
\includegraphics[scale=0.35]{F115}} 

\bigskip

\parbox[b][2cm][t]{130mm}{{\bf Ф11.6} В электрической цепи, схема которой приведена на рисунке, все элементы идеальные. Параметры элементов указаны на схеме. Ключ $K$ замыкают. В некоторый момент времени ток в цепи становится в $4$ раза меньше максимального.}
\hfill
\parbox[b][2cm][t]{50mm}{
\includegraphics[scale=0.35]{F116}} 
\begin{itemize}
\item[1)] Найти напряжение на катушке индуктивности $L$ в этот момент времени.
\item[2)] Чему равна скорость $P$ изменения энергии в катушке в этот момент времени?
\end{itemize}

\bigskip

\parbox[b][3cm][t]{145mm}{{\bf Ф11.7} На горизонтальную поверхность клиновидной пластинки из стекла по вертикали падает луч света. Расстояние от ребра клина до места падения луча равно $L=10$~см. Угол при вершине клина $\alpha = 0{,}2$~рад (см. рис.). На каком расстоянии от места падения луч <<выйдет>> из горизонтальной поверхности пластинки?


\textit{Указание.} Угол $\alpha$ можно считать малым, так что $\sin \alpha \approx \alpha$.}
\hfill
\parbox[b][3cm][t]{35mm}{
\includegraphics[scale=0.35]{F117}} 





\newpage
\begin{center}
Выездная олимпиада по математике\hfill\textit{\bf 8 класс}
\hrule\medskip
{\bf (из варианта исключается одна из задач 8.2, 8.4)}
\end{center}



{\bf М8.1} Числа $1$, $2$, $3$, $4$, $5$, $6$, $7$, $8$, $9$ разбили на две группы. Произведение чисел в первой группе равно $A$, а во второй группе~---~$B$. Известно, что число $C = \dfrac{A}{B}$~---~целое. Какое наименьшее значение может иметь число $C$? 

\bigskip

{\bf М8.2} Известно, что $3$ калача и $1$ баранка стоят дороже $100$ рублей, а $1$ калач и $13$ баранок также стоят дороже $100$ рублей. Верно ли, что $1$ калач и $4$ баранки стоят дороже $50$ рублей?

\bigskip

{\bf М8.3} Пусть $ABCD$ и $DEFG$~---~параллелограммы такие, что точка $D$ лежит на отрезке $AG$, точка $E$~---~на отрезке $DC$, и при этом $AB = DG = 2AD = 2DE$. Пусть $M$~---~середина отрезка $DG$. Докажите, что $CG$~---~биссектриса угла $MCF$.

\bigskip

{\bf М8.4} Можно ли расположить по кругу числа $0$, $1$, $2$, $\dotsc$, $9$ так, чтобы сумма любых трех последовательных чисел была не больше $14$?

\bigskip

{\bf М8.5} На столе лежат $300$ монет. Петя, Вася и Толя играют в следующую игру. Они ходят по очереди в следующем порядке: Петя, Вася, Толя, Петя, Вася, Толя, и т.~д. За один ход Петя может взять со стола $1$, $2$, $3$ или $4$ монеты, Вася~---~$1$ или $2$ монеты, а Толя~---~тоже $1$ или $2$ монеты. Могут ли Вася и Толя договориться так, что, как бы ни играл Петя, кто-то из них двоих заберет со стола последнюю монету? 

\bigskip
\newpage
\begin{center}
Выездная олимпиада по математике\hfill\textit{\bf 9 класс}
\hrule\medskip
{\bf (из варианта исключается одна из задач 9.2, 9.4)}
\end{center}

{\bf М9.1} Найдите какое-нибудь натуральное число $N$ такое, что если к нему прибавить его наибольший делитель, отличный от $N$, то получится $1212$.

\bigskip

{\bf М9.2} Верно ли, что при любых $a$ и $b$ хотя бы одно из уравнений $x^2 - 2ax + ab = 0$ и $x^2 - 2bx + ab = 0$ имеет решение?

\bigskip

{\bf М9.3} Можно ли расставить все натуральные числа от $1$ до $1000$ по кругу так, чтобы сумма любых трех подряд идущих чисел была простым числом?

\bigskip

{\bf М9.4} Действительные числа $a$ и $b$ удовлетворяют неравенствам $a \geq 1$, $b \geq 1$, $a+b \leq 2d$. Докажите, что $\sqrt{(a-1)(b+1)} + \sqrt{(b-1)(a+1)} < \sqrt{4d^2-2}$.

\bigskip

{\bf М9.5} Пусть $AL$~---~биссектриса остроугольного треугольника $ABC$, а $\omega$~---~описанная около него окружность. Обозначим через $P$ точку пересечения продолжения высоты $BH$ треугольника $ABC$ с окружностью $\omega$. Докажите, что если $\angle BLA = \angle BAC$, то $BP = CP$. 

\bigskip

\newpage
\begin{center}
Выездная олимпиада по математике\hfill\textit{\bf 10 класс}
\hrule\medskip
{\bf (из варианта исключается одна из задач 10.2, 10.3)}
\end{center}

{\bf М10.1} Второй, первый и третий члены арифметической прогрессии с ненулевой разностью образуют в указанном порядке геометрическую прогрессию. Найдите ее знаменатель.

\bigskip

{\bf М10.2} Пусть сумма чисел $a$, $b$, $c$ положительна. Докажите, что уравнение $a(x-b)(x-c) + b(x\hm-a)(x-c) + c(x-a)(x-b) = 0$ имеет хотя бы один действительный корень.

\bigskip

{\bf М10.3} Натуральное число $n$ таково, что $(n+1)!+(n+1)$ делится на $n! + n$. Какие значения может принимать $n$?

\bigskip

{\bf М10.4} Пусть $A = \sqrt{12+\sqrt{12+\sqrt{12+\dotsc}}} + \sqrt{20+\sqrt{20+\sqrt{20+\dotsc}}}$ (в каждом слагаемом $2019$ корней). Что больше: $A$ или $9$?

\bigskip

{\bf М10.5} Внутри окружности расположен пятиугольник $ABCDE$, у которого все стороны одинаковы. Каждая сторона пятиугольника продолжена до пересечения с окружностью. Продолжения лучей $AB$, $BC$, $CD$, $DE$, $EA$ окрашены в синий цвет, остальных лучей~---~в красный цвет. Докажите, что сумма длин всех красных отрезков равна сумме длин всех синих отрезков.

\bigskip

\newpage
\begin{center}
Выездная олимпиада по математике\hfill\textit{\bf 11 класс}
\hrule\medskip
{\bf (из варианта исключается одна из задач 11.2, 11.4)}
\end{center}

{\bf М11.1} Ненулевые числа $a$, $b$ и $c$ таковы, что числа $a(b-c)$, $b(c-a)$ и $c(a-b)$, записанные в указанном порядке, образуют арифметическую прогрессию. Докажите, что тогда и числа $a(b^{2109}-c^{2109})$, $b(c^{2109}-a^{2109})$ и $c(a^{2109}-b^{2109})$ также образуют арифметическую прогрессию. 

\bigskip

{\bf М11.2} Известно, что трехчлен $x^2+ax+b$ с положительными коэффициентами имеет два корня. Докажите, что для любого натурального $n$ можно один из коэффициентов трехчлена увеличить на $n$, какой-то другой на $n+1$, а третий оставить без изменения так, что получившийся трехчлен тоже будет иметь два корня.

\bigskip

{\bf М11.3} Существуют ли различные простые числа $m$, $n$ такие, что $m!+m$ делится на $n!+n$?

\bigskip

{\bf М11.4} Пусть $A = \sqrt{20+\sqrt{20+\sqrt{20+\dotsc}}} + \sqrt{30+\sqrt{30+\sqrt{30+\dotsc}}} + \sqrt{42+\sqrt{42+\sqrt{42+\dotsc}}}$ (в каждом слагаемом $2019$ корней). Что больше: $A$ или $18$?

\bigskip

{\bf М11.5} В треугольной пирамиде $SABC$ на ребре $SB$ выбраны точки $M$, $N$, а на ребре $SC$~---~точки $K$, $L$. Оказалось, что точки $A$, $M$, $N$, $K$, $L$ лежат на одной сфере, а объемы пирамид $ASKN$ и $ASML$ равны. Докажите, что $KL = MN$.

\bigskip

\newpage

\textrm{ }

\vspace{16 cm}

\begin{center}
{\bf Межвузовский центр воспитания и развития талантливой молодежи в
области естественно-математических наук «Физтех-Центр»}
\end{center}

Сборник подготовили:

\smallskip

Солоднев С.~А., Останин П.~А., Гаврилов Ю.~А., Диких, Д.~А., Зарубин И.~Е., Поминов С.~С., Щербина Е.~Н, Мукин Т.~В., Шомполов И.~Г., Трушин В.~Б., Черкасова Е.~К., Сидорова И.~Е., Подлипский О.~К., Агаханов Н.~Х., Усков В.~В., Плис В.~И., Чивилёв В.~И., Шеронов А.~А., Юрьев Ю.~В.

\smallskip

Под общей редакцией Шомполова И.~Г.

\smallskip

Компьютерный набор Останин П.~А.

\smallskip

Материалы данного конкурса доступны для свободного некоммерческого использования (при использовании ссылка на источник обязательна).

\smallskip

\copyright\textbf{ }Московский физико-технический институт (государственный университет), 2018-2019.

\end{document} 